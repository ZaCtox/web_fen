\chapter{Resultados}
\label{ch:results}

Este capítulo presenta los resultados obtenidos durante el desarrollo del proyecto, organizados en cuatro etapas principales: (1) levantamiento de requerimientos, (2) diseño del sistema, (3) implementación de la plataforma web, y (4) resultados finales y métricas. El análisis se basó en instrumentos aplicados al Director Administrativo y a la asistente de postgrado de la Facultad de Economía y Negocios de la Universidad de Talca, cuyas respuestas validaron la pertinencia de una solución tecnológica para optimizar la gestión operativa.

%=============================================================================
\section{Resultados del levantamiento de requerimientos}
%=============================================================================

\subsection{Resultados de la entrevista al Director Administrativo}

Durante la entrevista con el Director Administrativo se identificaron los siguientes puntos críticos en la gestión operativa actual:

\begin{itemize}
    \item La planificación de actividades y asignación de salas se gestiona manualmente, con alta carga operativa concentrada en una sola persona (asistente administrativa).
    \item Se reportaron dificultades frecuentes durante jornadas presenciales: cortes de luz, agua y fallas en la infraestructura tecnológica.
    \item Existe alta dependencia del trabajo de la asistente para mantener la operatividad durante los días sábado, principal día de clases de postgrado.
    \item Se valoró positivamente la propuesta de implementar un calendario interactivo y fichas de sala con información sobre equipamiento, limpieza y estado general.
    \item Se validó la pertinencia de la solución tecnológica, destacando la importancia de su usabilidad y adaptabilidad a la realidad operativa de la facultad.
\end{itemize}

El Director Administrativo mencionó además la necesidad de contemplar distintos perfiles de usuario y considerar al personal administrativo y de TI en futuras validaciones del sistema.

\subsection{Resultados de la encuesta a la asistente de postgrado}

La encuesta fue respondida por la asistente administrativa encargada de coordinar los programas de postgrado. Entre los hallazgos más relevantes se identificaron:

\begin{itemize}
    \item La planificación de cursos y asignación de salas se realiza mediante \textbf{correo electrónico y hojas de cálculo Excel}, sin apoyo de sistemas formales o institucionales.
    \item Las incidencias técnicas o logísticas se anotan en una bitácora personal en papel, sin trazabilidad institucional ni registro sistemático.
    \item Los principales problemas detectados en clases híbridas fueron: dificultades de audio, conexión intermitente a Zoom y fallas en la grabación de sesiones.
    \item Se resuelven entre 6 y 10 incidencias por mes, aunque no se lleva un registro formal de resolución ni tiempos de respuesta.
    \item La falta de una plataforma integrada afecta significativamente la eficiencia del trabajo administrativo.
\end{itemize}

La encuestada otorgó una calificación de \textbf{5 sobre 5} a la necesidad de implementar una solución tecnológica que centralice y automatice los procesos actuales.

\subsection{Síntesis de necesidades detectadas}

A partir del análisis de ambos instrumentos, se identificaron las siguientes necesidades clave que deben ser abordadas por la plataforma:

\begin{enumerate}
    \item Centralizar la planificación y visualización del uso de salas, cursos e incidencias en un solo sistema.
    \item Reducir la dependencia de herramientas externas como Excel y correo electrónico.
    \item Facilitar la comunicación entre actores mediante notificaciones y visibilidad compartida de información.
    \item Incluir trazabilidad institucional de incidencias mediante descripciones, imágenes y responsables.
    \item Considerar perfiles diferenciados por rol (administrador, director, asistente, estudiante).
    \item Organizar la información por períodos académicos (trimestres) y años de ingreso (cohortes).
    \item Generar reportes automáticos para mejorar la toma de decisiones administrativas.
    \item Proporcionar acceso público a información relevante (calendario, personal, documentos).
\end{enumerate}

\subsection{Relación entre necesidades detectadas y solución propuesta}

La Tabla~\ref{tab:necesidades-funcionalidades} muestra la correspondencia directa entre los problemas identificados durante el levantamiento y las funcionalidades implementadas en la plataforma web.

\begin{table}[H]
\centering
\caption{Relación entre necesidades detectadas y funcionalidades de la plataforma}
\label{tab:necesidades-funcionalidades}
\begin{tabular}{p{6.5cm} p{8.5cm}}
\toprule
\textbf{Necesidad / Problema detectado} & \textbf{Funcionalidad implementada} \\
\midrule
Planificación manual de actividades y asignación de salas. 
& \textbf{Módulo de gestión de salas:} registro, edición y asignación de aulas por período académico, con disponibilidad visible en tiempo real y prevención de conflictos. \\[0.3cm]

Incidencias frecuentes durante jornadas presenciales sin registro formal. 
& \textbf{Sistema de incidencias (tickets):} registro estructurado de problemas con imágenes, descripción, prioridad, estado y generación de reportes en PDF. \\[0.3cm]

Alta dependencia del trabajo de una sola persona para la operación. 
& \textbf{Sistema multinivel de roles:} distribución de responsabilidades entre administradores, directores, asistentes y personal TI, con permisos diferenciados. \\[0.3cm]

Necesidad de calendario accesible para estudiantes, docentes y administrativos. 
& \textbf{Calendario académico interactivo:} visualización de actividades con filtros por programa, sala, período y tipo de evento. Versión pública y administrativa. \\[0.3cm]

Falta de información detallada sobre salas y equipamiento. 
& \textbf{Ficha completa de sala:} incluye capacidad, ubicación, equipamiento disponible (proyector, pizarra, audio, HDMI), descripción y observaciones. \\[0.3cm]

Ausencia de trazabilidad institucional en resolución de incidencias. 
& \textbf{Sistema de seguimiento:} cada incidencia registra quién reporta, quién resuelve, tiempo de resolución, comentarios y estado actual. Historial completo de cambios. \\[0.3cm]

Uso de Excel y correos para gestión de cursos y documentos. 
& \textbf{Repositorio centralizado:} gestión de cursos, programas, documentos oficiales e informes con control de visibilidad, búsqueda y almacenamiento en la nube. \\[0.3cm]

Necesidad de reportes para toma de decisiones administrativas. 
& \textbf{Dashboard administrativo y sistema de analytics:} estadísticas visuales sobre uso de salas, incidencias, accesos al sistema y métricas de rendimiento operativo. \\
\bottomrule
\end{tabular}
\\[0.2cm]
\textit{Fuente: Elaboración propia a partir de entrevista, encuesta y análisis de requerimientos.}
\end{table}

%=============================================================================
\section{Resultados del diseño del sistema}
%=============================================================================

\subsection{Arquitectura del sistema: Patrón Modelo-Vista-Controlador}

La plataforma fue desarrollada siguiendo el patrón arquitectónico \textbf{Modelo-Vista-Controlador (MVC)}, implementado de manera nativa por el framework Laravel. Este patrón permite la separación de responsabilidades en tres componentes fundamentales:

\begin{itemize}
    \item \textbf{Modelo (Model):} representa la capa de datos y lógica de negocio. Los modelos interactúan con la base de datos mediante Eloquent ORM, definiendo la estructura de datos, relaciones entre entidades (uno a muchos, muchos a muchos), reglas de validación y métodos de consulta personalizados.
    
    \item \textbf{Vista (View):} constituye la capa de presentación que el usuario visualiza e interactúa. Laravel utiliza el motor de plantillas Blade, que combina HTML con directivas PHP. Las vistas renderizan la interfaz, muestran datos recibidos desde controladores y capturan entradas del usuario mediante formularios.
    
    \item \textbf{Controlador (Controller):} actúa como intermediario entre Modelos y Vistas, orquestando la lógica de la aplicación. Recibe peticiones HTTP, valida datos de entrada, invoca métodos del modelo, prepara datos para las vistas y retorna respuestas (HTML, JSON, PDF o redirecciones).
\end{itemize}

\textbf{Ventajas del patrón MVC en el proyecto:}

\begin{itemize}
    \item \textit{Separación de responsabilidades:} la lógica de negocio está separada de la presentación, facilitando el trabajo en equipo. Por ejemplo, cambios en la interfaz de usuario (vistas) no requieren modificar la lógica de acceso a datos (modelos).
    
    \item \textit{Reutilización de código:} un mismo modelo puede ser usado por controladores web y API. El modelo \texttt{ClaseSesion} es utilizado tanto por el controlador web \texttt{ClaseSesionController} como por el controlador API \texttt{Api\textbackslash ClaseController}, evitando duplicación de lógica de negocio.
    
    \item \textit{Facilidad de testing:} permite tests unitarios para modelos (validación de relaciones y métodos), tests de integración para controladores (verificación de endpoints) y tests de feature para flujos completos (simulación de casos de uso reales).
    
    \item \textit{Escalabilidad:} agregar nuevas funcionalidades no afecta código existente. La incorporación del módulo de Analytics no requirió modificar los módulos de Clases o Incidencias, solo agregar nuevos modelos y controladores.
\end{itemize}

\subsubsection{Ejemplo de flujo MVC completo en el sistema}

Para ilustrar el funcionamiento del patrón MVC, se describe el flujo completo cuando un usuario solicita visualizar el calendario académico:

\begin{enumerate}
    \item \textbf{Petición HTTP:} El usuario accede a la URL \texttt{/calendario} mediante el navegador.
    
    \item \textbf{Enrutamiento:} Laravel procesa la ruta definida en \texttt{routes/web.php}, identificando que debe invocar el método \texttt{calendario()} del controlador \texttt{EventController}.
    
    \item \textbf{Controlador:} El método \texttt{calendario()} realiza las siguientes acciones:
    \begin{itemize}
        \item Obtiene el período académico actual mediante el modelo \texttt{Period}.
        \item Obtiene los filtros desde la petición (año de ingreso, programa, trimestre).
        \item Consulta las sesiones de clase mediante el modelo \texttt{ClaseSesion}, aplicando eager loading para evitar el problema N+1:
        \begin{verbatim}
        ClaseSesion::with(['clase.course.magister', 'clase.room'])
            ->whereHas('clase.period', function($q) use ($filtros) {
                // aplicar filtros
            })
            ->get();
        \end{verbatim}
        \item Transforma los datos al formato requerido por FullCalendar (JSON).
        \item Prepara variables para la vista: \texttt{\$sesiones}, \texttt{\$periodoActual}, \texttt{\$magisters}, \texttt{\$rooms}.
    \end{itemize}
    
    \item \textbf{Vista:} El controlador retorna la vista Blade \texttt{calendario/index.blade.php}, pasando las variables preparadas. La vista:
    \begin{itemize}
        \item Renderiza el HTML base con Tailwind CSS.
        \item Incluye los filtros dinámicos con Alpine.js.
        \item Inicializa FullCalendar con JavaScript, consumiendo los datos de sesiones.
        \item Aplica estilos adaptativos según el dispositivo (responsive).
    \end{itemize}
    
    \item \textbf{Respuesta HTTP:} El navegador recibe el HTML completo y lo renderiza al usuario, quien visualiza el calendario interactivo con todas las sesiones de clase.
    
    \item \textbf{Interacción adicional (AJAX):} Si el usuario aplica un filtro o hace clic en un evento:
    \begin{itemize}
        \item JavaScript envía una petición AJAX al controlador.
        \item El controlador consulta nuevamente los modelos con los nuevos filtros.
        \item Retorna respuesta JSON con los datos actualizados.
        \item JavaScript actualiza la vista sin recargar la página completa.
    \end{itemize}
\end{enumerate}

Este flujo se repite de manera similar en todos los módulos del sistema, garantizando consistencia arquitectónica y mantenibilidad del código.

\subsection{Diseño de la base de datos}

La base de datos se implementó utilizando \textbf{SQLite} para desarrollo y \textbf{MySQL} para producción. El diseño relacional contempla 18 tablas principales organizadas en módulos funcionales:

\subsubsection{Módulo de usuarios y autenticación}

\begin{itemize}
    \item \textbf{users:} almacena información de usuarios (nombre, email, contraseña hasheada, rol, año de ingreso, foto de perfil).
    \item \textbf{sessions:} gestión de sesiones web activas.
    \item \textbf{personal\_access\_tokens:} tokens de autenticación para API (Laravel Sanctum).
\end{itemize}

\subsubsection{Módulo académico}

\begin{itemize}
    \item \textbf{magisters:} programas de magíster con información institucional, director, color identificador y orden de visualización.
    \item \textbf{periods:} trimestres académicos con fechas, vinculados a programas y cohortes específicas (año de ingreso).
    \item \textbf{courses:} asignaturas con profesor, créditos SCT, tipo (obligatorio/electivo) y prerequisitos.
    \item \textbf{clases:} instancias de cursos en trimestres específicos, con horario, sala, modalidad (presencial/online/híbrida) y enlace Zoom.
    \item \textbf{clase\_sesiones:} sesiones individuales de cada clase con fecha, estado, grabación y bloques de descanso (coffee/lunch breaks).
\end{itemize}

\subsubsection{Módulo de infraestructura}

\begin{itemize}
    \item \textbf{rooms:} salas físicas con capacidad, ubicación, equipamiento y descripción.
\end{itemize}

\subsubsection{Módulo de personal}

\begin{itemize}
    \item \textbf{staff:} personal académico y administrativo con cargo, contacto, biografía y foto.
\end{itemize}

\subsubsection{Módulo de incidencias}

\begin{itemize}
    \item \textbf{incidents:} sistema de tickets con número, prioridad, estado, sala afectada, imagen adjunta y responsables de reporte y resolución.
    \item \textbf{incident\_logs:} historial de cambios en cada incidencia.
\end{itemize}

\subsubsection{Módulo de documentos y comunicaciones}

\begin{itemize}
    \item \textbf{informes:} documentos oficiales con tipo, visibilidad pública/privada y almacenamiento en nube.
    \item \textbf{novedades:} noticias y anuncios con tipo, visibilidad por roles, fecha de expiración y acciones personalizadas.
    \item \textbf{daily\_reports:} reportes diarios en PDF.
    \item \textbf{report\_entries:} entradas individuales de reportes con sala, tipo, severidad y foto.
\end{itemize}

\subsubsection{Módulo de emergencias y analytics}

\begin{itemize}
    \item \textbf{emergencies:} alertas críticas activas con tipo, mensaje y expiración.
    \item \textbf{page\_views:} registro de accesos a la plataforma con IP, navegador, sesión y timestamp.
\end{itemize}

\subsubsection{Relaciones entre entidades}

El modelo de datos implementa relaciones jerárquicas que reflejan la estructura académica:

\begin{itemize}
    \item \textit{Jerarquía académica:} Magister $\rightarrow$ Periods y Courses $\rightarrow$ Clases $\rightarrow$ ClaseSesiones.
    \item \textit{Asignación de recursos:} Room $\rightarrow$ Clases, Incidents, ReportEntries.
    \item \textit{Creación y autoría:} User $\rightarrow$ Incidents, Informes, Novedades, DailyReports.
    \item \textit{Trazabilidad:} Incident $\rightarrow$ IncidentLogs; DailyReport $\rightarrow$ ReportEntries.
\end{itemize}

Se aplicaron restricciones de integridad referencial mediante claves foráneas con acciones \texttt{ON DELETE CASCADE} y \texttt{ON DELETE SET NULL}, índices únicos y validaciones a nivel de base de datos para garantizar consistencia.

\subsubsection{Ejemplos de consultas y optimizaciones}

El sistema implementa diversas estrategias de optimización para garantizar rendimiento eficiente:

\paragraph{Eager Loading para evitar problema N+1}

Una de las optimizaciones más importantes es el uso de \textbf{eager loading} en consultas que involucran relaciones. Por ejemplo, al listar clases con sus cursos, programas y salas:

\begin{verbatim}
// Sin optimización (problema N+1 - múltiples consultas)
$clases = Clase::all(); // 1 consulta
foreach($clases as $clase) {
    echo $clase->course->nombre;      // N consultas adicionales
    echo $clase->course->magister->nombre; // N más consultas
}

// Con eager loading (optimizado - 3 consultas totales)
$clases = Clase::with(['course.magister', 'room', 'period'])
    ->get();
\end{verbatim}

Esta técnica reduce drásticamente el número de consultas a la base de datos, mejorando significativamente el tiempo de respuesta en listados complejos.

\paragraph{Índices estratégicos}

Se definieron índices compuestos para optimizar las consultas más frecuentes:

\begin{itemize}
    \item \texttt{(clase\_id, fecha)} en \texttt{clase\_sesiones}: garantiza unicidad y acelera búsquedas por clase y fecha.
    \item \texttt{(page\_type, visited\_at)} en \texttt{page\_views}: optimiza análisis de accesos por tipo de página en rangos temporales.
    \item \texttt{(magister\_id, anio\_ingreso)} en \texttt{periods}: facilita filtrado de períodos por programa y cohorte.
    \item \texttt{(estado, created\_at)} en \texttt{incidents}: acelera listados de incidencias pendientes ordenadas cronológicamente.
\end{itemize}

\paragraph{Consultas agregadas eficientes}

El módulo de analytics utiliza consultas agregadas optimizadas para calcular métricas sin procesar grandes volúmenes de datos en PHP:

\begin{verbatim}
// Tiempo promedio de resolución de incidencias (en horas)
$tiempoPromedio = Incident::where('estado', 'resuelta')
    ->whereNotNull('resuelta_en')
    ->selectRaw('AVG(TIMESTAMPDIFF(HOUR, created_at, resuelta_en)) 
                 as promedio')
    ->value('promedio');

// Distribución de incidencias por sala
$porSala = Incident::select('room_id', DB::raw('count(*) as total'))
    ->groupBy('room_id')
    ->with('room:id,name')
    ->orderByDesc('total')
    ->get();
\end{verbatim}

\paragraph{Scopes personalizados en modelos}

Los modelos implementan scopes reutilizables para filtros comunes:

\begin{verbatim}
// En el modelo PageView
public function scopeThisMonth($query) {
    return $query->whereYear('visited_at', now()->year)
                 ->whereMonth('visited_at', now()->month);
}

public function scopeCalendarioPublico($query) {
    return $query->where('page_type', 'calendario_publico');
}

// Uso en controlador
$accesos = PageView::thisMonth()
    ->calendarioPublico()
    ->count();
\end{verbatim}

Estas optimizaciones garantizan que el sistema mantenga buen rendimiento incluso con volúmenes crecientes de datos.

\subsection{Arquitectura tecnológica}

La arquitectura de tres capas separa las responsabilidades y permite escalabilidad:

\subsubsection{Backend (Capa de Aplicación)}

\begin{itemize}
    \item \textbf{Laravel 12} (PHP 8.2) como framework principal.
    \item \textbf{SQLite/MySQL} como sistema gestor de base de datos relacional.
    \item \textbf{Laravel Sanctum} para autenticación de API mediante tokens.
    \item \textbf{Cloudinary} para gestión y almacenamiento de archivos multimedia en la nube.
    \item \textbf{DomPDF} para generación dinámica de reportes en formato PDF.
\end{itemize}

\subsubsection{Frontend (Capa de Presentación)}

\begin{itemize}
    \item \textbf{Tailwind CSS} para diseño responsive y modo oscuro.
    \item \textbf{Alpine.js} para interactividad ligera sin frameworks pesados.
    \item \textbf{Vite} como bundler modular para optimización de assets.
    \item \textbf{FullCalendar} para visualización avanzada de calendarios académicos.
    \item \textbf{SweetAlert2} para alertas y notificaciones amigables.
\end{itemize}

\subsubsection{API RESTful (Capa de Integración)}

\begin{itemize}
    \item Endpoints públicos para consulta de información académica sin autenticación.
    \item Endpoints protegidos con token Sanctum para operaciones CRUD.
    \item Arquitectura que permite consumo desde aplicaciones móviles o servicios externos.
\end{itemize}

%=============================================================================
\section{Resultados de la implementación}
%=============================================================================

\subsection{Módulos funcionales desarrollados}

La plataforma se estructuró en 15 módulos independientes pero interconectados. La Tabla~\ref{tab:modulos-tecnico} presenta un resumen técnico de cada módulo.

\begin{table}[H]
\centering
\caption{Resumen técnico de módulos funcionales implementados}
\label{tab:modulos-tecnico}
\small
\begin{tabular}{p{3.2cm} p{5.5cm} p{1.5cm} p{1.5cm} p{1.8cm}}
\toprule
\textbf{Módulo} & \textbf{Funcionalidad principal} & \textbf{Tablas BD} & \textbf{Vistas} & \textbf{Complejidad} \\
\midrule
Autenticación y Usuarios & Registro, login, recuperación de contraseña, gestión de perfiles y roles & 3 & 8 & Alta \\[0.2cm]

Magisters & Administración de programas académicos, cohortes y datos institucionales & 1 & 4 & Media \\[0.2cm]

Períodos y Trimestres & Gestión de períodos académicos con fechas, prevención de traslapes & 1 & 4 & Media \\[0.2cm]

Cursos y Asignaturas & Catálogo de cursos con profesor, créditos, tipo y prerequisitos & 1 & 4 & Media \\[0.2cm]

Clases y Sesiones & Programación de clases, sesiones individuales, bloques de descanso & 2 & 6 & Alta \\[0.2cm]

Calendario Académico & Visualización interactiva con vistas mensual, semanal y agenda & 0 & 2 & Alta \\[0.2cm]

Salas y Recursos & Gestión de aulas, capacidad, equipamiento y disponibilidad & 1 & 4 & Baja \\[0.2cm]

Personal (Staff) & Directorio de equipo académico y administrativo con vista pública & 1 & 4 & Baja \\[0.2cm]

Incidencias & Sistema de tickets con prioridad, estado, responsables y métricas & 2 & 5 & Media \\[0.2cm]

Informes y Documentos & Repositorio con control de visibilidad y almacenamiento en nube & 1 & 4 & Media \\[0.2cm]

Novedades y Comunicaciones & Canal institucional con expiración automática y segmentación & 1 & 4 & Media \\[0.2cm]

Reportes Diarios & Documentación diaria con generación de PDF y adjuntos & 2 & 3 & Media \\[0.2cm]

Emergencias & Alertas críticas visibles en toda la plataforma & 1 & 1 & Baja \\[0.2cm]

Analytics y Estadísticas & Monitoreo de uso, accesos y métricas operativas & 1 & 1 & Media \\[0.2cm]

API RESTful & Integración externa con endpoints públicos y protegidos & --- & --- & Alta \\
\bottomrule
\end{tabular}
\\[0.2cm]
\textit{Fuente: Elaboración propia basada en documentación técnica del sistema.}
\end{table}

\subsubsection{Descripción de módulos clave}

\paragraph{Gestión de Clases y Sesiones}

Este módulo permite programar cursos en trimestres específicos, asignando horarios, salas físicas y modalidad de dictado (presencial, online o híbrida). Genera automáticamente sesiones individuales basadas en el horario de clase y contempla bloques de descanso (coffee breaks y lunch breaks) en clases extensas.

\textit{Impacto:} optimiza la asignación de recursos, reduce conflictos de programación y centraliza la información de grabaciones y enlaces virtuales.

\paragraph{Calendario Académico Interactivo}

Visualiza todas las actividades académicas en formato calendario con tres vistas: mensual, semanal y de agenda. Integra información de programas, cursos, salas y sesiones, mostrando eventos con código de colores según programa. Incluye versión pública (sin autenticación) y administrativa (con información completa).

\textit{Impacto:} mejora la comunicación entre administrativos, docentes y estudiantes, garantizando acceso actualizado al calendario académico.

\paragraph{Sistema de Incidencias}

Sistema de tickets para reportar problemas técnicos o logísticos. Cada incidencia incluye número de ticket, prioridad (baja, media, alta, crítica), estado (pendiente, en proceso, resuelta, cerrada), sala afectada, imagen adjunta y responsables de reporte y resolución. Genera historial completo de cambios.

\textit{Impacto:} permite trazabilidad institucional, mejora tiempos de respuesta y genera evidencia para análisis posteriores.

\paragraph{Sistema de Analytics y Monitoreo}

Registra métricas de uso de la plataforma mediante un middleware de tracking: número de accesos, páginas más visitadas, tiempo promedio de resolución de incidencias y porcentaje de utilización del calendario académico. Presenta dashboard con gráficos interactivos y KPIs principales.

\textit{Impacto:} proporciona visión global del desempeño operativo y apoya la toma de decisiones basada en datos.

\subsection{API RESTful para integración móvil}

Se desarrolló una API completa que replica la funcionalidad de la plataforma web, permitiendo su consumo desde aplicaciones móviles o servicios externos. Características principales:

\begin{itemize}
    \item \textbf{Endpoints públicos:} consulta de calendarios, cursos, programas, personal, informes y novedades sin autenticación.
    \item \textbf{Endpoints protegidos:} operaciones CRUD para usuarios autenticados mediante tokens Sanctum.
    \item \textbf{Filtros avanzados:} por año de ingreso, programa, trimestre, tipo, fecha y búsqueda de texto.
    \item \textbf{Paginación:} para optimizar transferencia de datos en conexiones móviles.
    \item \textbf{Formato JSON estandarizado:} respuestas consistentes con códigos HTTP correctos.
\end{itemize}

La API pasó por una auditoría exhaustiva que garantizó sincronización 100\% entre funcionalidad web y API, corrección de rutas duplicadas y validación de permisos.

\subsubsection{Ejemplos de uso de la API}

A continuación se presentan ejemplos concretos de consumo de la API desde aplicaciones externas:

\paragraph{Autenticación y obtención de token}

\begin{verbatim}
POST /api/login
Content-Type: application/json

{
    "email": "usuario@utalca.cl",
    "password": "contraseña_segura"
}

Respuesta:
{
    "access_token": "1|abc123...",
    "token_type": "Bearer",
    "user": {
        "id": 15,
        "name": "Juan Pérez",
        "email": "usuario@utalca.cl",
        "rol": "estudiante"
    }
}
\end{verbatim}

\paragraph{Consulta de calendario académico (endpoint público)}

\begin{verbatim}
GET /api/public/events?anio_ingreso=2024&start=2024-10-01&end=2024-10-31

Respuesta:
[
    {
        "id": 1234,
        "title": "Economía I",
        "start": "2024-10-05T09:00:00",
        "end": "2024-10-05T16:30:00",
        "programa": "Magíster en Gestión",
        "programa_color": "#3B82F6",
        "sala": "A301",
        "modalidad": "presencial",
        "profesor": "Dr. Carlos Silva"
    },
    ...
]
\end{verbatim}

\paragraph{Creación de incidencia (endpoint protegido)}

\begin{verbatim}
POST /api/incidents
Authorization: Bearer 1|abc123...
Content-Type: application/json

{
    "titulo": "Proyector no funciona",
    "descripcion": "El proyector de la sala A301 no enciende",
    "sala": "A301",
    "prioridad": "alta",
    "room_id": 5
}

Respuesta:
{
    "message": "Incidencia creada exitosamente",
    "incident": {
        "id": 89,
        "nro_ticket": "INC-2024-089",
        "titulo": "Proyector no funciona",
        "estado": "pendiente",
        "created_at": "2024-10-22T14:30:00.000000Z"
    }
}
\end{verbatim}

\paragraph{Filtrado avanzado de cursos}

\begin{verbatim}
GET /api/public/courses?anio_ingreso=2024&magister_id=3&search=econom

Respuesta:
{
    "data": [
        {
            "id": 45,
            "nombre": "Economía Internacional",
            "codigo": "ECON-502",
            "profesor": "Dra. María González",
            "sct": 10,
            "tipo": "obligatorio",
            "magister": {
                "id": 3,
                "nombre": "Magíster en Gestión"
            }
        },
        ...
    ],
    "meta": {
        "current_page": 1,
        "total": 5,
        "per_page": 15
    }
}
\end{verbatim}

\subsubsection{Validaciones y manejo de errores}

La API implementa validaciones robustas y retorna códigos HTTP apropiados:

\begin{itemize}
    \item \textbf{200 OK:} operación exitosa con datos.
    \item \textbf{201 Created:} recurso creado exitosamente.
    \item \textbf{400 Bad Request:} datos de entrada inválidos con detalles de errores de validación.
    \item \textbf{401 Unauthorized:} token ausente o inválido.
    \item \textbf{403 Forbidden:} usuario autenticado pero sin permisos para la acción.
    \item \textbf{404 Not Found:} recurso no existe.
    \item \textbf{422 Unprocessable Entity:} validación de negocio fallida.
    \item \textbf{500 Internal Server Error:} error del servidor con mensaje genérico.
\end{itemize}

Ejemplo de respuesta de error de validación:

\begin{verbatim}
POST /api/incidents (con datos inválidos)

Respuesta (422):
{
    "message": "Los datos proporcionados no son válidos.",
    "errors": {
        "titulo": ["El campo título es obligatorio."],
        "prioridad": ["La prioridad debe ser: baja, media, alta o critica."]
    }
}
\end{verbatim}

\subsection{Sistema de autenticación y autorización}

La seguridad se implementó mediante un sistema de roles y permisos multinivel con seis roles principales:

\begin{enumerate}
    \item \textbf{Administrador:} acceso total al sistema, configuraciones globales.
    \item \textbf{Director Administrativo:} gestión operativa completa, reportes, incidencias, personal.
    \item \textbf{Director de Programa:} gestión de su magíster, cursos, clases y estudiantes.
    \item \textbf{Asistente de Postgrado:} soporte administrativo, gestión de clases y calendario.
    \item \textbf{Estudiante:} consulta de calendario, cursos, informes y novedades.
    \item \textbf{Invitado:} solo lectura de información pública.
\end{enumerate}

\textbf{Mecanismos de seguridad implementados:}

\begin{itemize}
    \item Autenticación web con Laravel Breeze (sesiones).
    \item Autenticación API con Laravel Sanctum (tokens).
    \item Middleware de verificación de roles en todas las rutas protegidas.
    \item Hashing de contraseñas con bcrypt.
    \item Protección CSRF en formularios.
\end{itemize}

\subsection{Testing y aseguramiento de calidad}

Se implementó una estrategia de testing automatizado utilizando Pest PHP:

\begin{itemize}
    \item \textbf{Tests unitarios:} validación de lógica de negocio en modelos.
    \item \textbf{Tests de integración:} verificación de endpoints API.
    \item \textbf{Tests de feature:} simulación de flujos completos de usuario.
\end{itemize}

El código se mantiene con estándares de calidad mediante Laravel Pint (formateo PSR-12) y PHPStan (análisis estático). Scripts batch automatizados facilitan la ejecución rápida de suites de tests.

\subsubsection{Ejemplos de tests implementados}

\paragraph{Test unitario de modelo}

Validación de relaciones y métodos del modelo \texttt{Magister}:

\begin{verbatim}
test('un magister tiene muchos cursos', function () {
    $magister = Magister::factory()->create();
    $courses = Course::factory(3)->create([
        'magister_id' => $magister->id
    ]);
    
    expect($magister->courses)->toHaveCount(3);
    expect($magister->courses->first())->toBeInstanceOf(Course::class);
});

test('un magister puede calcular total de créditos SCT', function () {
    $magister = Magister::factory()->create();
    Course::factory()->create(['magister_id' => $magister->id, 'sct' => 10]);
    Course::factory()->create(['magister_id' => $magister->id, 'sct' => 15]);
    
    expect($magister->totalCreditosSCT())->toBe(25);
});
\end{verbatim}

\paragraph{Test de integración de API}

Verificación de endpoint de creación de incidencias con autenticación:

\begin{verbatim}
test('usuario autenticado puede crear incidencia', function () {
    $user = User::factory()->create(['rol' => 'administrador']);
    $room = Room::factory()->create();
    
    $response = $this->actingAs($user)->postJson('/api/incidents', [
        'titulo' => 'Proyector averiado',
        'descripcion' => 'No enciende',
        'room_id' => $room->id,
        'prioridad' => 'alta'
    ]);
    
    $response->assertStatus(201)
        ->assertJson([
            'message' => 'Incidencia creada exitosamente'
        ]);
        
    $this->assertDatabaseHas('incidents', [
        'titulo' => 'Proyector averiado',
        'user_id' => $user->id
    ]);
});

test('usuario no autenticado no puede crear incidencia', function () {
    $response = $this->postJson('/api/incidents', [
        'titulo' => 'Test'
    ]);
    
    $response->assertStatus(401);
});
\end{verbatim}

\paragraph{Test de feature completo}

Simulación de flujo completo de usuario creando y resolviendo una incidencia:

\begin{verbatim}
test('flujo completo de gestión de incidencia', function () {
    // 1. Usuario reporta incidencia
    $reportador = User::factory()->create(['rol' => 'estudiante']);
    
    $this->actingAs($reportador)->post('/incidencias', [
        'titulo' => 'Luz no funciona',
        'descripcion' => 'Sala A301 sin luz',
        'sala' => 'A301',
        'prioridad' => 'media'
    ]);
    
    $incidencia = Incident::latest()->first();
    expect($incidencia->estado)->toBe('pendiente');
    
    // 2. Administrador asigna y resuelve
    $admin = User::factory()->create(['rol' => 'administrador']);
    
    $this->actingAs($admin)->patch("/incidencias/{$incidencia->id}", [
        'estado' => 'resuelta',
        'comentario' => 'Fusible reemplazado'
    ]);
    
    $incidencia->refresh();
    expect($incidencia->estado)->toBe('resuelta');
    expect($incidencia->resolved_by)->toBe($admin->id);
    expect($incidencia->resuelta_en)->not->toBeNull();
    
    // 3. Se registra en el log
    expect($incidencia->logs)->toHaveCount(1);
});
\end{verbatim}

\subsubsection{Cobertura de tests y CI/CD}

El proyecto cuenta con scripts automatizados para ejecutar la suite completa de tests:

\begin{itemize}
    \item \texttt{run-tests.bat}: ejecuta todos los tests con reporte de cobertura.
    \item \texttt{test-api-simple.ps1}: tests específicos de endpoints API.
    \item \texttt{test-analytics.bat}: tests del módulo de analytics.
    \item \texttt{test-publicos.bat}: tests de endpoints públicos sin autenticación.
\end{itemize}

Los tests se ejecutan automáticamente antes de cada commit mediante hooks de Git, garantizando que no se introduzcan regresiones en el código.

\subsection{Diseño de interfaz y experiencia de usuario}

El diseño siguió principios de UX/UI modernos, priorizando claridad, accesibilidad y consistencia visual:

\begin{itemize}
    \item \textbf{Diseño responsive:} adaptación automática a móviles, tablets y escritorio.
    \item \textbf{Modo oscuro:} soporte completo en toda la plataforma.
    \item \textbf{Componentes reutilizables:} consistencia visual mediante Blade components.
    \item \textbf{Navegación intuitiva:} menús contextuales según rol del usuario.
    \item \textbf{Feedback visual inmediato:} validaciones en tiempo real y alertas amigables con SweetAlert2.
    \item \textbf{Accesibilidad:} uso de etiquetas semánticas y contraste adecuado.
\end{itemize}

El uso de Tailwind CSS permitió desarrollo ágil sin sacrificar personalización, mientras que Alpine.js proporcionó interactividad ligera sin complejidad de frameworks JavaScript pesados.

\subsubsection{Principios de diseño aplicados}

El desarrollo de la interfaz siguió principios reconocidos de interacción humano-computador:

\begin{itemize}
    \item \textbf{Ley de Jakob:} los usuarios pasan la mayor parte del tiempo en otros sitios web, por lo que se aplicaron patrones de diseño familiares (navegación lateral, breadcrumbs, modales, formularios estándar) para reducir la curva de aprendizaje.
    
    \item \textbf{Ley de Fitts:} los elementos interactivos importantes (botones de acción principal, enlaces frecuentes) se posicionaron en áreas de fácil acceso y se les otorgó mayor tamaño para facilitar la interacción, especialmente en dispositivos móviles.
    
    \item \textbf{Ley de Miller (7±2):} la información se agrupó en bloques manejables. Por ejemplo, los filtros del módulo de incidencias se organizan en categorías colapsables, y las tablas muestran información priorizada sin sobrecargar visualmente al usuario.
    
    \item \textbf{Principio de proximidad (Gestalt):} elementos relacionados se agrupan visualmente mediante tarjetas, secciones y espaciado consistente, facilitando la comprensión de relaciones entre datos.
    
    \item \textbf{Jerarquía visual:} uso estratégico de tamaño, color y contraste para guiar la atención del usuario hacia elementos importantes. Los estados críticos (incidencias urgentes, emergencias activas) utilizan colores llamativos (rojo, naranja) para captar atención inmediata.
\end{itemize}

\subsubsection{Componentes reutilizables implementados}

Se desarrollaron componentes Blade reutilizables que garantizan consistencia visual en toda la plataforma:

\begin{itemize}
    \item \texttt{x-app-layout}: estructura base de todas las páginas autenticadas (navegación, sidebar, footer).
    \item \texttt{x-guest-layout}: estructura para páginas públicas sin autenticación.
    \item \texttt{x-action-button}: botones de acción con estilos consistentes y estados (hover, disabled).
    \item \texttt{x-hci-breadcrumb}: navegación de migas de pan adaptativa.
    \item \texttt{x-input}, \texttt{x-select}, \texttt{x-textarea}: campos de formulario con validación visual incorporada.
    \item \texttt{x-modal}: ventanas modales accesibles y responsive.
    \item \texttt{x-alert}: mensajes de éxito, error, advertencia e información con íconos distintivos.
\end{itemize}

Estos componentes aceleran el desarrollo de nuevas funcionalidades al proporcionar bloques de construcción pre-diseñados y testeados.

%=============================================================================
\section{Resultados finales y métricas del sistema}
%=============================================================================

\subsection{Métricas de desarrollo}

El proceso de desarrollo resultó en una plataforma con las siguientes características cuantificables:

\begin{table}[H]
\centering
\caption{Métricas técnicas del sistema implementado}
\label{tab:metricas-desarrollo}
\begin{tabular}{lr}
\toprule
\textbf{Componente} & \textbf{Cantidad} \\
\midrule
Controladores API & 16 \\
Controladores Web & 33 \\
Tablas de base de datos & 18 \\
Migraciones & 89 \\
Vistas Blade & 151 \\
Modelos Eloquent & 18 \\
Documentos técnicos & 47 \\
Módulos funcionales & 15 \\
Roles de usuario & 6 \\
Endpoints API públicos & 15 \\
Endpoints API protegidos & $\sim$60 \\
\bottomrule
\end{tabular}
\\[0.2cm]
\textit{Fuente: Métricas extraídas del repositorio del proyecto.}
\end{table}

\subsection{Interconexión entre módulos}

Los módulos no funcionan de manera aislada, sino que están interconectados siguiendo la lógica académica y operativa de la facultad. La jerarquía de relaciones implementada es:

\begin{itemize}
    \item \textit{Jerarquía académica:} Magisters $\rightarrow$ Periods y Courses $\rightarrow$ Clases $\rightarrow$ ClaseSesiones $\rightarrow$ Calendario.
    \item \textit{Asignación de recursos:} Rooms $\rightarrow$ Clases, Incidents, ReportEntries.
    \item \textit{Creación y autoría:} Users $\rightarrow$ Novedades, Informes, DailyReports, Incidents.
    \item \textit{Monitoreo transversal:} Analytics monitorea accesos, incidencias, calendario y reportes.
\end{itemize}

\subsection{Evaluación de la solución implementada}

La plataforma desarrollada responde directamente a las necesidades detectadas durante el levantamiento de requerimientos:

\begin{itemize}
    \item \textbf{Funcional:} cubre todas las necesidades identificadas mediante 15 módulos interconectados que abordan desde la planificación académica hasta la gestión de incidencias y generación de reportes.
    
    \item \textbf{Escalable:} arquitectura modular que permite agregar funcionalidades sin afectar código existente. La incorporación de nuevos módulos (como Analytics) no requirió modificar módulos previos, validando la separación de responsabilidades.
    
    \item \textbf{Mantenible:} código limpio siguiendo estándares PSR-12 y buenas prácticas de Laravel. El uso consistente del patrón MVC y componentes reutilizables facilita la comprensión y modificación del código por desarrolladores futuros.
    
    \item \textbf{Segura:} sistema robusto de autenticación y autorización con 6 roles diferenciados. Implementa protección CSRF, hashing bcrypt, validación de tokens Sanctum y middleware de verificación de permisos en todas las rutas sensibles.
    
    \item \textbf{Documentada:} 47 documentos técnicos que facilitan mantenimiento y extensión futura, incluyendo guías de API, documentación de módulos, diagramas de arquitectura y manuales de testing.
    
    \item \textbf{Testeada:} suite de tests automatizados con Pest PHP que valida funcionalidad crítica mediante tests unitarios (modelos), de integración (API) y de feature (flujos completos).
    
    \item \textbf{Integrable:} API RESTful completa con 16 controladores, 75+ endpoints y filtros avanzados, permitiendo consumo desde aplicaciones móviles con autenticación mediante tokens.
    
    \item \textbf{Optimizada:} implementa eager loading, índices estratégicos, consultas agregadas y scopes reutilizables para garantizar rendimiento eficiente incluso con volúmenes crecientes de datos.
    
    \item \textbf{Usable:} interfaz diseñada siguiendo principios de HCI (Leyes de Jakob, Fitts, Miller), con diseño responsive, modo oscuro, componentes accesibles y retroalimentación visual inmediata.
\end{itemize}

\subsection{Impacto operativo medible}

La implementación de la plataforma generó mejoras cuantificables en la gestión operativa:

\begin{table}[H]
\centering
\caption{Comparación antes y después de la implementación}
\label{tab:impacto-operativo}
\begin{tabular}{p{5cm} p{4.5cm} p{4.5cm}}
\toprule
\textbf{Aspecto} & \textbf{Antes} & \textbf{Después} \\
\midrule
Planificación de salas & Excel manual con duplicación de información & Sistema centralizado en tiempo real con prevención de conflictos \\[0.2cm]

Registro de incidencias & Bitácora en papel sin trazabilidad & Sistema de tickets digital con historial completo y métricas \\[0.2cm]

Acceso al calendario & Archivos PDF estáticos enviados por correo & Calendario interactivo público 24/7 con filtros \\[0.2cm]

Gestión de documentos & Carpetas en servidor local & Repositorio centralizado en nube con control de visibilidad \\[0.2cm]

Comunicación institucional & Correos masivos manuales & Sistema de novedades con segmentación automática \\[0.2cm]

Tiempo de consulta de información & 5-10 minutos (buscar en Excel/correos) & Inmediato (búsqueda en tiempo real) \\[0.2cm]

Generación de reportes & Manual con Excel (30-60 min) & Automática con exportación PDF (< 1 min) \\[0.2cm]

Dependencia operativa & Alta (1 persona clave) & Distribuida (múltiples roles con permisos) \\
\bottomrule
\end{tabular}
\\[0.2cm]
\textit{Fuente: Comparación basada en entrevistas y observación del uso del sistema.}
\end{table}

\subsection{Logros técnicos destacados}

Entre los logros técnicos más significativos se encuentran:

\begin{enumerate}
    \item \textbf{Sincronización Web-API 100\%:} La auditoría exhaustiva garantizó que todos los filtros, validaciones y funcionalidades de las vistas web estén disponibles en la API, permitiendo desarrollar aplicaciones móviles con paridad funcional completa.
    
    \item \textbf{Sistema de Analytics en tiempo real:} Mediante middleware de tracking transparente, se recopilan métricas de uso sin afectar rendimiento, permitiendo análisis de comportamiento y toma de decisiones basada en datos.
    
    \item \textbf{Gestión avanzada de bloques horarios:} Implementación de coffee breaks y lunch breaks en sesiones de clase extensas, con visualización diferenciada en calendario y cálculo automático de horarios.
    
    \item \textbf{Exportación dinámica de reportes PDF:} Generación on-the-fly de documentos profesionales con Blade templates y DomPDF, incluyendo logotipos, filtros aplicados y datos estadísticos.
    
    \item \textbf{Sistema de búsqueda global:} Implementación de búsqueda transversal que permite encontrar información en múltiples entidades (cursos, clases, salas, personal) desde un solo punto de acceso.
    
    \item \textbf{Gestión multinivel de roles:} Sistema flexible que permite asignar permisos granulares según el rol del usuario, facilitando la distribución de responsabilidades administrativas.
\end{enumerate}

\subsection{Conclusión}

El desarrollo e implementación de la plataforma web consolidó un sistema integral que responde efectivamente a las problemáticas detectadas en la gestión operativa de los programas de postgrado de la Facultad de Economía y Negocios. 

La combinación del patrón arquitectónico MVC, un diseño de base de datos relacional normalizado y optimizado, una API RESTful completa con validaciones robustas, y un sistema de roles multinivel, garantiza una solución técnicamente sólida, escalable y mantenible a largo plazo.

El proceso iterativo de desarrollo permitió validar cada módulo con usuarios finales, asegurando que las funcionalidades implementadas no solo cumplen con los requerimientos técnicos, sino que satisfacen las necesidades reales y cotidianas del personal administrativo, docentes y estudiantes. La plataforma logró:

\begin{itemize}
    \item Eliminar la dependencia de herramientas externas (Excel, correos) para gestión operativa.
    \item Centralizar información dispersa en un sistema único y accesible.
    \item Proporcionar trazabilidad institucional en procesos críticos (incidencias, reportes).
    \item Mejorar significativamente los tiempos de respuesta y consulta de información.
    \item Distribuir la carga operativa mediante roles y permisos diferenciados.
    \item Habilitar el acceso público a información relevante (calendario, personal, documentos).
    \item Generar métricas y reportes que apoyan la toma de decisiones administrativas.
\end{itemize}

Los resultados cuantitativos (16 controladores API, 89 migraciones, 151 vistas, 18 tablas, 15 módulos funcionales) y cualitativos (mejora en tiempos, eliminación de procesos manuales, mayor satisfacción de usuarios) validan que los objetivos del proyecto fueron alcanzados exitosamente.

La plataforma representa un avance significativo en la modernización tecnológica de la gestión académica y administrativa, estableciendo bases sólidas para futuras expansiones y mejoras continuas del sistema.

%=============================================================================
\section{Descripción detallada de módulos principales}
%=============================================================================

A partir de los requerimientos levantados en las entrevistas y encuestas, se desarrollaron 15 módulos funcionales dentro de la plataforma web (descritos en la Tabla~\ref{tab:modulos-tecnico}). Si bien todos contribuyen a la mejora de la gestión operativa de los programas de postgrado, cinco módulos destacan por su complejidad técnica, frecuencia de uso y relevancia estratégica en la resolución de las problemáticas identificadas.

Esta sección profundiza en la descripción técnica e implementación de estos cinco módulos principales, seleccionados en base a los siguientes criterios:

\begin{enumerate}
    \item \textbf{Impacto directo en necesidades críticas:} abordan los problemas más urgentes detectados durante el levantamiento de requerimientos (planificación de salas, gestión de incidencias, visualización de calendario).
    \item \textbf{Complejidad técnica significativa:} requieren integración de múltiples tecnologías (FullCalendar, Alpine.js, Chart.js) y manejo avanzado de estados y eventos.
    \item \textbf{Uso transversal por múltiples roles:} son utilizados tanto por personal administrativo como por directores, docentes y estudiantes.
    \item \textbf{Interconexión con otros módulos:} actúan como núcleos que integran información de múltiples entidades del sistema.
\end{enumerate}

Los módulos seleccionados para descripción detallada son: (1) Gestión de Salas, (2) Calendario Académico Interactivo, (3) Bitácora de Incidencias, (4) Dashboard Administrativo, y (5) Gestión de Programas y Cursos. Cada descripción incluye estructura funcional, tecnologías empleadas, interfaz de usuario e impacto operativo.

La estructura modular de la plataforma facilita su mantenimiento y escalabilidad, permitiendo incorporar nuevas funcionalidades en el futuro sin afectar la estabilidad del sistema principal.

%-----------------------------------------------------------------------------
\subsection{Módulo de Gestión de Salas}
%-----------------------------------------------------------------------------

El módulo de Gestión de Salas constituye una de las principales funcionalidades de la plataforma, permitiendo administrar de forma centralizada la información de los espacios físicos disponibles en la Facultad de Economía y Negocios. Su objetivo principal es optimizar la planificación académica y operativa, asegurando la trazabilidad, disponibilidad y correcta utilización de las salas asignadas a los distintos programas de postgrado.

Este módulo se desarrolló utilizando \textbf{Laravel Blade}, \textbf{Tailwind CSS} y \textbf{Alpine.js}, tecnologías que permiten un manejo dinámico de los datos y una experiencia fluida e intuitiva para el usuario. A través de su diseño responsive y visualmente coherente con el resto del sistema, facilita tanto la gestión administrativa como la consulta de información.

\subsubsection{Estructura y funcionalidades principales}

El módulo se compone de dos vistas principales:

\paragraph{a) Listado de Salas}

Esta vista muestra un listado general de todas las salas registradas, con la posibilidad de realizar búsquedas, filtrar resultados y acceder a acciones directas como editar o eliminar registros. Las funcionalidades principales incluyen:

\begin{itemize}
    \item \textbf{Búsqueda en tiempo real:} mediante el uso de Alpine.js, permite filtrar las salas por nombre o ubicación sin recargar la página.
    
    \item \textbf{Registro de nuevas salas:} el botón ``Agregar Sala'' dirige al formulario de creación, donde el usuario puede ingresar datos como nombre, ubicación, capacidad y equipamiento disponible.
    
    \item \textbf{Acciones rápidas:} cada registro cuenta con botones para:
    \begin{itemize}
        \item Ver la ficha técnica de la sala.
        \item Consultar sus clases asignadas.
        \item Editar o eliminar la sala directamente desde la tabla.
    \end{itemize}
    
    \item \textbf{Diseño accesible y adaptativo:} la interfaz utiliza componentes reutilizables y elementos visuales (íconos, colores y sombras) para mejorar la claridad en la navegación.
\end{itemize}

En caso de no encontrarse resultados, el sistema despliega un estado vacío personalizado, invitando al usuario a limpiar filtros o realizar una nueva búsqueda.

\paragraph{b) Detalle de Sala}

Esta vista ofrece una ficha técnica completa con toda la información relevante de una sala específica. Se organiza mediante pestañas (tabs) que permiten alternar entre dos secciones principales:

\begin{itemize}
    \item \textbf{Ficha técnica:} muestra los datos generales de la sala como ubicación, capacidad y descripción, además de un listado visual de características y equipamiento (calefacción, pizarra, televisor, computador, energía eléctrica, aseo disponible, etc.). Cada ítem se representa mediante íconos y colores que indican su disponibilidad, entregando una vista clara del estado de cada sala.
    
    \item \textbf{Clases asignadas:} presenta las sesiones o clases que se realizan en la sala seleccionada, permitiendo aplicar filtros por programa, día, año o trimestre. El usuario puede identificar rápidamente las clases por color y acceder a su detalle mediante un enlace directo. Los filtros se implementaron con Alpine.js, lo que permite cambiar las combinaciones sin recargar la página.
\end{itemize}

Esta estructura mejora significativamente la usabilidad y rapidez de consulta, ya que centraliza toda la información operativa de una sala en una sola interfaz.

\subsubsection{Impacto funcional}

El módulo de gestión de salas permitió eliminar el uso de planillas Excel y registros manuales, integrando la información en una base de datos única y accesible desde cualquier dispositivo con conexión a internet. Gracias a su vinculación con el módulo de clases y el calendario académico, es posible evitar conflictos de horarios y disponer de una vista global del uso de espacios físicos, favoreciendo la coordinación entre los distintos programas.

%-----------------------------------------------------------------------------
\subsection{Módulo de Calendario Académico Interactivo}
%-----------------------------------------------------------------------------

El módulo de calendario académico representa el núcleo visual y operativo del sistema, permitiendo gestionar y visualizar de manera interactiva las actividades académicas asociadas a los distintos programas. Este componente integra tanto las clases programadas automáticamente como los eventos manuales creados por los usuarios administrativos, consolidando toda la información en una interfaz centralizada y dinámica.

Su implementación se desarrolló utilizando \textbf{FullCalendar} como biblioteca principal de visualización, junto con \textbf{Laravel Blade}, \textbf{JavaScript moderno} y \textbf{Tailwind CSS}, logrando un equilibrio entre rendimiento, estética y usabilidad. El módulo es totalmente interactivo y responsive, adaptándose a distintos tamaños de pantalla y roles de usuario dentro de la plataforma.

\subsubsection{Estructura y funcionamiento general}

El calendario se carga dentro de la vista principal mediante renderizado dinámico de eventos. El script de administración (\texttt{calendar-admin.js}) gestiona toda la lógica de interacción: obtención de datos desde el backend, renderizado, filtrado, creación, edición y eliminación de eventos.

Entre sus principales características se destacan:

\begin{itemize}
    \item \textbf{Visualización por períodos académicos:} el calendario muestra las clases y eventos organizados por trimestres académicos, con actualización automática del texto del período actual (por ejemplo, ``Trimestre II del año 2025''). Esta información se obtiene dinámicamente desde el backend mediante peticiones a la API de períodos.
    
    \item \textbf{Vistas adaptativas:} dependiendo del dispositivo, el calendario alterna automáticamente entre vista semanal (\texttt{timeGridWeek}) en equipos de escritorio y vista de lista (\texttt{listWeek}) en móviles, mejorando la legibilidad y navegación.
    
    \item \textbf{Gestión de eventos académicos:} los usuarios con permisos administrativos pueden crear nuevos eventos directamente sobre el calendario mediante la selección de rangos de tiempo. Se abre un modal personalizado donde se ingresan datos como título, programa, sala, fechas y horas. Estos datos se envían mediante peticiones AJAX con protección CSRF, utilizando los métodos POST, PUT o DELETE según la acción (crear, editar o eliminar).
    
    \item \textbf{Diferenciación visual:} cada evento se muestra con un color asociado al programa correspondiente. Además, se distinguen los eventos manuales (creados por el usuario) y los eventos pasados mediante clases CSS especiales que modifican su apariencia visual. Los eventos de descanso como ``COFFEE BREAK'' o ``ALMUERZO'' se muestran con íconos y estilos diferenciados para facilitar la identificación visual.
    
    \item \textbf{Filtros avanzados:} se incluyen filtros dinámicos por:
    \begin{itemize}
        \item Programa de magíster
        \item Sala física
        \item Año de ingreso (cohorte)
        \item Año académico
        \item Trimestre específico
    \end{itemize}
    Estos filtros permiten personalizar la vista del calendario y actualizar los eventos en tiempo real sin recargar la página.
    
    \item \textbf{Modal de detalles enriquecido:} al hacer clic sobre un evento, se despliega un modal informativo con todos los detalles: título, profesor, programa, modalidad, sala, horario, enlaces de Zoom y grabación (si aplica). Este modal incluye botones de acción para editar o eliminar el evento, dependiendo de su tipo y permisos del usuario.
    
    \item \textbf{Validación visual y retroalimentación:} el módulo incorpora alertas y mensajes de confirmación mediante \textbf{SweetAlert2}, ofreciendo una interacción clara y moderna con el usuario. Además, maneja los posibles errores de red o validación con mensajes explicativos y consistentes.
\end{itemize}

\subsubsection{Impacto funcional}

El módulo de calendario académico logró automatizar la gestión y visualización de la carga académica, reduciendo significativamente el tiempo de planificación y mejorando la coordinación entre programas, docentes y salas. Gracias a su integración con los módulos de clases y salas, se evita la superposición de horarios y se garantiza una distribución eficiente de los recursos.

Asimismo, el uso de herramientas modernas como FullCalendar y modales interactivos ha permitido una experiencia más fluida para los usuarios administrativos, quienes pueden modificar la planificación en tiempo real sin recurrir a hojas de cálculo o correos internos.

%-----------------------------------------------------------------------------
\subsection{Módulo de Bitácora de Incidencias}
%-----------------------------------------------------------------------------

El módulo de Bitácora de Incidencias permite registrar, gestionar y hacer seguimiento de los problemas o eventos que afectan la infraestructura y los servicios del sistema. Su propósito es mantener un historial centralizado, trazable y confiable de todas las incidencias ocurridas, garantizando una gestión más transparente, ordenada y eficiente dentro de la Facultad.

Este módulo fue diseñado bajo principios de usabilidad, jerarquía visual y consistencia, aplicando criterios de la \textbf{Ley de Jakob} (coherencia con el resto del sistema), la \textbf{Ley de Miller} (agrupación y priorización de información) y la \textbf{Ley de Fitts} (facilitación de interacción mediante botones accesibles). Su interfaz está construida con \textbf{Laravel Blade}, \textbf{Alpine.js} y \textbf{Tailwind CSS}, integrando además componentes reutilizables de diseño que aportan uniformidad y claridad visual.

\subsubsection{Estructura funcional}

El módulo está compuesto por dos vistas principales:

\paragraph{a) Listado de Incidencias}

Esta vista presenta una tabla dinámica y filtrable con todas las incidencias registradas en el sistema. Desde aquí, el usuario puede buscar, filtrar, exportar y acceder al detalle de cada incidencia.

\textbf{Principales características:}

\begin{itemize}
    \item \textbf{Filtros avanzados dinámicos:} implementados mediante Alpine.js, permiten combinar múltiples criterios de búsqueda sin recargar la página. Los filtros incluyen:
    \begin{itemize}
        \item Estado de la incidencia (pendiente, en revisión, resuelta, no resuelta)
        \item Sala afectada
        \item Programa asociado
        \item Año académico, trimestre o años históricos
        \item Año de ingreso de los estudiantes vinculados
    \end{itemize}
    
    \item \textbf{Modo histórico:} permite visualizar incidencias fuera de los períodos académicos actuales, facilitando el análisis longitudinal. El sistema adapta automáticamente los filtros visibles según este modo, ocultando o mostrando los campos relevantes.
    
    \item \textbf{Exportación de datos:} el usuario puede descargar el listado filtrado en formato PDF. Este reporte incluye el logotipo institucional, fecha de generación, usuario responsable, filtros aplicados y un resumen de incidencias (totales, pendientes, resueltas, etc.). El documento es generado dinámicamente desde Blade con estilos CSS personalizados, garantizando un formato profesional y uniforme.
    
    \item \textbf{Diseño responsive e interactivo:} la tabla se adapta a distintos tamaños de pantalla y utiliza efectos visuales (hover y color) para mejorar la legibilidad. Cada fila es clicable, redirigiendo directamente al detalle de la incidencia seleccionada.
    
    \item \textbf{Estados visuales con íconos:} cada registro muestra un ícono distintivo según su estado (por ejemplo, reloj ⏳ para pendiente, check ✅ para resuelta), permitiendo una interpretación rápida y visual del estado general del sistema.
\end{itemize}

\paragraph{b) Detalle de Incidencia}

Esta vista presenta en profundidad la información asociada a una incidencia, dividida en secciones estructuradas y diferenciadas:

\begin{itemize}
    \item \textbf{Encabezado con acciones rápidas:} incluye botones fijos para volver al listado o eliminar la incidencia. Las acciones se controlan mediante roles, asegurando que solo usuarios autorizados puedan modificar o eliminar registros.
    
    \item \textbf{Información general y responsables:} muestra los datos principales como ID, número de ticket, sala afectada, estado actual, usuario creador y responsable de resolución. La disposición en tarjetas de dos columnas mejora la lectura y el orden visual.
    
    \item \textbf{Evidencia fotográfica:} si la incidencia contiene imágenes, se muestra una vista previa ampliable en modal, optimizando la presentación de evidencias visuales y permitiendo un análisis más completo del problema.
    
    \item \textbf{Actualización de estado:} los usuarios con permisos administrativos o técnicos pueden modificar el estado, registrar comentarios o asociar un número de ticket externo. Los campos están protegidos por validaciones que aseguran la integridad de la información.
    
    \item \textbf{Historial cronológico (timeline):} cada cambio de estado se almacena automáticamente, mostrando fecha, usuario y comentario asociado. Este registro contribuye a la trazabilidad y transparencia del proceso de resolución.
\end{itemize}

\subsubsection{Submódulo de Estadísticas de Incidencias}

Complementariamente, el sistema cuenta con una sección de análisis visual y métricas que permiten evaluar el desempeño del proceso de resolución de incidencias. Esta vista utiliza \textbf{Chart.js} para la generación de gráficos dinámicos e interactivos, y Alpine.js para la gestión de filtros sin recarga de página.

Los principales indicadores mostrados son:

\begin{itemize}
    \item Total de incidencias registradas
    \item Porcentaje de incidencias resueltas
    \item Promedio de tiempo de resolución (en horas)
    \item Tiempo promedio por estado (pendiente, en revisión, resuelta)
    \item Distribución de incidencias por sala, programa y trimestre académico
\end{itemize}

Los gráficos (barras, líneas y tipo doughnut) son adaptativos al modo oscuro o claro y cuentan con botones de descarga en formato PNG, permitiendo su uso en reportes o presentaciones institucionales. El cálculo de métricas se realiza mediante consultas agregadas con Eloquent ORM, garantizando que los datos estadísticos reflejen el estado real del sistema.

\subsubsection{Integración con otros módulos}

El módulo de incidencias se integra estrechamente con el sistema de salas y el gestor de roles de usuario, permitiendo:

\begin{itemize}
    \item Asociar las incidencias a una sala específica del sistema
    \item Controlar qué usuarios pueden crear, editar o resolver incidencias
    \item Generar métricas y reportes por trimestre, año o programa
    \item Visualizar el comportamiento histórico desde el módulo de estadísticas
\end{itemize}

\subsubsection{Impacto y resultados}

El Módulo de Bitácora de Incidencias ha mejorado significativamente la gestión y seguimiento de eventos dentro de la infraestructura académica. Gracias a su interfaz intuitiva, filtros inteligentes y herramientas de análisis, los usuarios administrativos pueden identificar patrones, priorizar mantenimiento y tomar decisiones basadas en datos reales e históricos.

Además, la integración con roles y la generación automática de reportes garantizan un proceso seguro, auditable y transparente, alineado con los objetivos de eficiencia y trazabilidad establecidos en el proyecto.

%-----------------------------------------------------------------------------
\subsection{Módulo de Dashboard Administrativo}
%-----------------------------------------------------------------------------

El módulo de Dashboard Administrativo constituye el punto de inicio y monitoreo principal del sistema, ofreciendo una visión general del estado operativo de la Facultad de Economía y Negocios. Su objetivo es proporcionar información resumida, actualizada y accesible a los distintos roles de usuario, facilitando la toma de decisiones y el control de las actividades diarias.

\subsubsection{Componentes del panel}

El dashboard está compuesto por diversas secciones interactivas:

\begin{itemize}
    \item \textbf{Bienvenida personalizada:} muestra el nombre, rol y fecha actual del usuario autenticado, generando una experiencia más cercana y personalizada.
    
    \item \textbf{Novedades del sistema:} presenta anuncios o actualizaciones relevantes publicadas por el equipo administrativo, permitiendo mantener informados a los usuarios sobre cambios o nuevas funcionalidades.
    
    \item \textbf{Estadísticas principales:} despliega indicadores clave como incidencias pendientes, emergencias activas, número de salas registradas y usuarios activos. Cada indicador se representa mediante tarjetas dinámicas que incorporan colores e íconos distintivos para facilitar la interpretación visual.
    
    \item \textbf{Próximas clases:} disponible para los usuarios con rol docente, muestra las próximas sesiones programadas con detalles de curso, fecha, horario y sala asignada.
    
    \item \textbf{Actividad reciente:} incluye un registro de las últimas incidencias y reportes diarios ingresados, lo que permite monitorear en tiempo real las acciones más recientes del sistema.
\end{itemize}

\subsubsection{Implementación técnica}

El diseño del panel utiliza \textbf{Laravel Blade} y \textbf{Tailwind CSS}, combinando una interfaz limpia, responsive y compatible con modo oscuro. Los datos se obtienen dinámicamente desde el controlador mediante variables estadísticas como \texttt{\$stats}, \texttt{\$novedades}, \texttt{\$proximasClases} y \texttt{\$actividadReciente}.

\subsubsection{Impacto funcional}

De esta manera, el módulo cumple una función clave en la plataforma: centralizar la información relevante y proporcionar una visión integral del funcionamiento diario de los programas de postgrado, contribuyendo al cumplimiento de los objetivos de eficiencia y control establecidos en el proyecto.

%-----------------------------------------------------------------------------
\subsection{Módulo de Gestión de Programas y Cursos}
%-----------------------------------------------------------------------------

El Módulo de Gestión de Programas y Cursos constituye una de las secciones más relevantes del sistema, ya que permite administrar la estructura académica de los programas de postgrado de la Facultad. Este módulo centraliza la información de los programas y sus cursos asociados, proporcionando herramientas de creación, edición, filtrado y visualización jerárquica de los contenidos académicos.

Desarrollado bajo el framework Laravel, y utilizando \textbf{Blade}, \textbf{Alpine.js} y \textbf{Tailwind CSS}, este módulo ofrece una interfaz limpia, moderna y coherente con el resto del sistema, optimizada para la gestión administrativa de datos y con soporte completo para modo oscuro, diseño responsive y componentes reutilizables.

\subsubsection{Estructura funcional}

El módulo se divide en dos vistas principales:

\paragraph{a) Listado de Programas}

Esta vista permite al usuario visualizar todos los programas registrados, con detalles administrativos como el nombre del programa, encargado, asistente, teléfono, correo, y la cantidad de cursos asociados y su total de créditos SCT.

Entre sus funcionalidades más destacadas se encuentran:

\begin{itemize}
    \item \textbf{Filtro por año de ingreso:} permite mostrar únicamente los programas vigentes o de años anteriores. El filtro actualiza automáticamente la lista sin necesidad de recargar la página.
    
    \item \textbf{Búsqueda dinámica:} mediante Alpine.js, los usuarios pueden buscar por nombre de programa en tiempo real, con visualización instantánea de resultados.
    
    \item \textbf{Gestión CRUD completa:} incluye botones para crear, editar y eliminar programas, todos acompañados de validaciones de rol y confirmaciones visuales. El sistema muestra advertencias cuando un programa tiene cursos asociados, para evitar eliminaciones accidentales.
    
    \item \textbf{Mensajes dinámicos e indicadores de contexto:} si el usuario consulta un año anterior, el sistema muestra una advertencia contextual (``⚠️ Mostrando programas de un Año de Ingreso Anterior''), ayudando a diferenciar períodos históricos.
    
    \item \textbf{Diseño visual adaptativo:} cada programa se representa dentro de una tarjeta con una línea lateral de color identificativo, y su contenido se ajusta automáticamente según el ancho del dispositivo.
\end{itemize}

\paragraph{b) Cursos por Programa}

Esta vista está orientada a mostrar de forma organizada los cursos asociados a cada programa, agrupados por año académico y trimestre, lo que facilita la planificación y revisión de la estructura curricular.

\textbf{Principales funcionalidades:}

\begin{itemize}
    \item \textbf{Agrupación jerárquica automática:} los cursos se agrupan dinámicamente por año y trimestre, permitiendo visualizar de manera clara la progresión académica. La vista utiliza tablas anidadas para mostrar los datos con mayor claridad.
    
    \item \textbf{Gestión de cursos:} desde cada encabezado de programa, el usuario puede agregar nuevos cursos, editar o eliminar los existentes. Los botones de acción se presentan con íconos e interacciones visuales, respetando los estándares de accesibilidad y usabilidad.
    
    \item \textbf{Identificación de requisitos:} cada curso muestra los requisitos académicos correspondientes, representados mediante etiquetas codificadas por color. Por ejemplo, los cursos que requieren ``Ingreso'' se muestran en verde, mientras que los que dependen de otro curso se indican en azul con el nombre del requisito.
    
    \item \textbf{Indicadores académicos:} para cada curso se presentan los créditos SCT, los requisitos y las acciones disponibles, lo que proporciona una visión completa y rápida del estado curricular del programa.
    
    \item \textbf{Desplegables interactivos:} cada programa puede expandirse o contraerse mediante encabezados con animación, implementados con JavaScript nativo. Esto permite una navegación más ordenada y una mejor gestión de la información sin sobrecargar la vista.
    
    \item \textbf{Modo responsive y accesible:} las tablas y botones se adaptan a distintos tamaños de pantalla, manteniendo la legibilidad en dispositivos móviles o de escritorio.
\end{itemize}

\subsubsection{Integración con otros módulos}

El Módulo de Gestión de Programas y Cursos se integra directamente con:

\begin{itemize}
    \item El módulo de Clases, permitiendo asignar cursos a períodos académicos específicos.
    \item El módulo de Salas, facilitando la planificación y asignación de espacios físicos según el curso y trimestre.
    \item El gestor de roles de usuario, que controla qué perfiles pueden crear, editar o eliminar información académica.
\end{itemize}

Esta interconexión permite mantener la coherencia de datos entre las distintas secciones del sistema y evita duplicidades en la administración académica.

\subsubsection{Impacto y resultados}

La implementación de este módulo ha simplificado considerablemente la gestión curricular y la planificación académica de los programas de postgrado. Gracias a la visualización jerárquica por año y trimestre, los usuarios administrativos pueden realizar un seguimiento integral de los cursos, detectar fácilmente ausencias o duplicidades y mantener actualizado el registro de créditos y requisitos.

Asimismo, el sistema proporciona una experiencia de usuario moderna, visualmente atractiva y funcional, que reduce los errores operativos y mejora la eficiencia en la administración de los programas.

%=============================================================================

